\documentclass[]{article}
\usepackage{lmodern}
\usepackage{amssymb,amsmath}
\usepackage{ifxetex,ifluatex}
\usepackage{fixltx2e} % provides \textsubscript
\ifnum 0\ifxetex 1\fi\ifluatex 1\fi=0 % if pdftex
  \usepackage[T1]{fontenc}
  \usepackage[utf8]{inputenc}
\else % if luatex or xelatex
  \ifxetex
    \usepackage{mathspec}
  \else
    \usepackage{fontspec}
  \fi
  \defaultfontfeatures{Ligatures=TeX,Scale=MatchLowercase}
\fi
% use upquote if available, for straight quotes in verbatim environments
\IfFileExists{upquote.sty}{\usepackage{upquote}}{}
% use microtype if available
\IfFileExists{microtype.sty}{%
\usepackage{microtype}
\UseMicrotypeSet[protrusion]{basicmath} % disable protrusion for tt fonts
}{}
\usepackage[margin=1in]{geometry}
\usepackage{hyperref}
\hypersetup{unicode=true,
            pdftitle={Gerber and Green Experiment},
            pdfborder={0 0 0},
            breaklinks=true}
\urlstyle{same}  % don't use monospace font for urls
\usepackage{graphicx,grffile}
\makeatletter
\def\maxwidth{\ifdim\Gin@nat@width>\linewidth\linewidth\else\Gin@nat@width\fi}
\def\maxheight{\ifdim\Gin@nat@height>\textheight\textheight\else\Gin@nat@height\fi}
\makeatother
% Scale images if necessary, so that they will not overflow the page
% margins by default, and it is still possible to overwrite the defaults
% using explicit options in \includegraphics[width, height, ...]{}
\setkeys{Gin}{width=\maxwidth,height=\maxheight,keepaspectratio}
\IfFileExists{parskip.sty}{%
\usepackage{parskip}
}{% else
\setlength{\parindent}{0pt}
\setlength{\parskip}{6pt plus 2pt minus 1pt}
}
\setlength{\emergencystretch}{3em}  % prevent overfull lines
\providecommand{\tightlist}{%
  \setlength{\itemsep}{0pt}\setlength{\parskip}{0pt}}
\setcounter{secnumdepth}{0}
% Redefines (sub)paragraphs to behave more like sections
\ifx\paragraph\undefined\else
\let\oldparagraph\paragraph
\renewcommand{\paragraph}[1]{\oldparagraph{#1}\mbox{}}
\fi
\ifx\subparagraph\undefined\else
\let\oldsubparagraph\subparagraph
\renewcommand{\subparagraph}[1]{\oldsubparagraph{#1}\mbox{}}
\fi

%%% Use protect on footnotes to avoid problems with footnotes in titles
\let\rmarkdownfootnote\footnote%
\def\footnote{\protect\rmarkdownfootnote}

%%% Change title format to be more compact
\usepackage{titling}

% Create subtitle command for use in maketitle
\newcommand{\subtitle}[1]{
  \posttitle{
    \begin{center}\large#1\end{center}
    }
}

\setlength{\droptitle}{-2em}

  \title{Gerber and Green Experiment}
    \pretitle{\vspace{\droptitle}\centering\huge}
  \posttitle{\par}
    \author{}
    \preauthor{}\postauthor{}
    \date{}
    \predate{}\postdate{}
  

\begin{document}
\maketitle

\section{Gerber and Green 2003}\label{gerber-and-green-2003}

\subsection{Exploratory data analysis}\label{exploratory-data-analysis}

In this experiment there is total number of 18 933 individuals from 6
different cities in the United States.

\subsubsection{The dependent and independent
variables}\label{the-dependent-and-independent-variables}

The dependent variable is whether the person voted in the 6 November
election in 2001. There are six independent variables; race, sex, age,
party affiliation, turnout in the 2000 election and turnout in the 1999
election. The treatment indicator variable

\subsubsection{Weird values}\label{weird-values}

\begin{verbatim}
##    Min. 1st Qu.  Median    Mean 3rd Qu.    Max.    NA's 
##   18.00   31.00   42.00   45.25   56.00 2001.00    2487
\end{verbatim}

\includegraphics{GerberGreen_Intro_files/figure-latex/unnamed-chunk-2-1.pdf}

\begin{verbatim}
##              city              precinct   zip race party sex  age turf
## 13846 MINNEAPOLIS MINNEAPOLIS W- 6 P- 8 55407                2001  103
##       voted01 voted00 voted99 family famsize represen reached other goaway
## 13846       1       0       0   4506       1        1       0     0      0
##       nothome bad cant nothing contact treatmen primary
## 13846       0   0    0       0       0        0       0
\end{verbatim}

\subsubsection{Missing data}\label{missing-data}

There is missing data in the experimental files - we need to recode them
to the NA format to further inspect them.

\includegraphics{GerberGreen_Intro_files/figure-latex/unnamed-chunk-4-1.pdf}

\begin{verbatim}
## [1] "Raleigh"
\end{verbatim}

\includegraphics{GerberGreen_Intro_files/figure-latex/missing values per city-1.pdf}

\begin{verbatim}
## [1] "Bridgeport"
\end{verbatim}

\includegraphics{GerberGreen_Intro_files/figure-latex/missing values per city-2.pdf}

\begin{verbatim}
## [1] "DETROIT"
\end{verbatim}

\includegraphics{GerberGreen_Intro_files/figure-latex/missing values per city-3.pdf}

\begin{verbatim}
## [1] "COLUMBUS"
\end{verbatim}

\includegraphics{GerberGreen_Intro_files/figure-latex/missing values per city-4.pdf}

\begin{verbatim}
## [1] "MINNEAPOLIS"
\end{verbatim}

\includegraphics{GerberGreen_Intro_files/figure-latex/missing values per city-5.pdf}

\begin{verbatim}
## [1] "ST PAUL"
\end{verbatim}

\includegraphics{GerberGreen_Intro_files/figure-latex/missing values per city-6.pdf}

From these plots we conclude that Columbus is not useful at all.

St Paul and Minneapolis only useful together. I will try this.

Detroit, Raleigh, Bridgeport.

\subsubsection{The rationale for putting Minneapolis and St Paul
together}\label{the-rationale-for-putting-minneapolis-and-st-paul-together}

\begin{itemize}
\item
  Same state
\item
  Same sampling composition
\item
  x Variables to be used: age, voted00, voted99
\end{itemize}

\emph{Dealing with the one weird individual who is 2001 years old}

\subsubsection{Detroit, Raleigh,
Bridgeport}\label{detroit-raleigh-bridgeport}

\begin{itemize}
\item
  Bridgeport seems a bit different; very low turnout. Different
  composition and so on.
\item
  Detroit, Raleigh is a stretch but let's try.
\end{itemize}

\subsection{Understanding the differences in background covariates in
Minnesota}\label{understanding-the-differences-in-background-covariates-in-minnesota}

\includegraphics{GerberGreen_Intro_files/figure-latex/Density plot age Min-1.pdf}

\begin{verbatim}
##    Min. 1st Qu.  Median    Mean 3rd Qu.    Max. 
##  0.0000  1.0000  1.0000  0.8174  1.0000  1.0000
\end{verbatim}

\begin{verbatim}
##    Min. 1st Qu.  Median    Mean 3rd Qu.    Max. 
##  0.0000  0.0000  1.0000  0.5389  1.0000  1.0000
\end{verbatim}

\includegraphics{GerberGreen_Intro_files/figure-latex/unnamed-chunk-7-1.pdf}

\subsubsection{Rescale the variables}\label{rescale-the-variables}

\subsubsection{Making prediction in
Minnesota}\label{making-prediction-in-minnesota}

First, we prepare the data. Let \(D = 0\) be Minneapolis and \(D = 1\)
St Paul.

Now we run causal Match just once to see

\includegraphics{GerberGreen_Intro_files/figure-latex/unnamed-chunk-10-1.pdf}

Now we obtain the final predictions:

\begin{verbatim}
## [1] "Building trees ..."
## [1] "Tree 1"
## [1] 2
## [1] "CT"
## [1] "Tree 2"
## [1] 2
## [1] "CT"
## [1] "Tree 3"
## [1] 2
## [1] "CT"
## [1] "Tree 4"
## [1] 2
## [1] "CT"
## [1] "Tree 5"
## [1] 2
## [1] "CT"
## [1] "Tree 6"
## [1] 2
## [1] "CT"
## [1] "Tree 7"
## [1] 2
## [1] "CT"
## [1] "Tree 8"
## [1] 2
## [1] "CT"
## [1] "Tree 9"
## [1] 2
## [1] "CT"
## [1] "Tree 10"
## [1] 2
## [1] "CT"
## [1] "Tree 11"
## [1] 2
## [1] "CT"
## [1] "Tree 12"
## [1] 2
## [1] "CT"
## [1] "Tree 13"
## [1] 2
## [1] "CT"
## [1] "Tree 14"
## [1] 2
## [1] "CT"
## [1] "Tree 15"
## [1] 2
## [1] "CT"
## [1] "Tree 16"
## [1] 2
## [1] "CT"
## [1] "Tree 17"
## [1] 2
## [1] "CT"
## [1] "Tree 18"
## [1] 2
## [1] "CT"
## [1] "Tree 19"
## [1] 2
## [1] "CT"
## [1] "Tree 20"
## [1] 2
## [1] "CT"
## [1] "Tree 21"
## [1] 2
## [1] "CT"
## [1] "Tree 22"
## [1] 2
## [1] "CT"
## [1] "Tree 23"
## [1] 2
## [1] "CT"
## [1] "Tree 24"
## [1] 2
## [1] "CT"
## [1] "Tree 25"
## [1] 2
## [1] "CT"
## [1] "Tree 26"
## [1] 2
## [1] "CT"
## [1] "Tree 27"
## [1] 2
## [1] "CT"
## [1] "Tree 28"
## [1] 2
## [1] "CT"
## [1] "Tree 29"
## [1] 2
## [1] "CT"
## [1] "Tree 30"
## [1] 2
## [1] "CT"
## [1] "Tree 31"
## [1] 2
## [1] "CT"
## [1] "Tree 32"
## [1] 2
## [1] "CT"
## [1] "Tree 33"
## [1] 2
## [1] "CT"
## [1] "Tree 34"
## [1] 2
## [1] "CT"
## [1] "Tree 35"
## [1] 2
## [1] "CT"
## [1] "Tree 36"
## [1] 2
## [1] "CT"
## [1] "Tree 37"
## [1] 2
## [1] "CT"
## [1] "Tree 38"
## [1] 2
## [1] "CT"
## [1] "Tree 39"
## [1] 2
## [1] "CT"
## [1] "Tree 40"
## [1] 2
## [1] "CT"
## [1] "Tree 41"
## [1] 2
## [1] "CT"
## [1] "Tree 42"
## [1] 2
## [1] "CT"
## [1] "Tree 43"
## [1] 2
## [1] "CT"
## [1] "Tree 44"
## [1] 2
## [1] "CT"
## [1] "Tree 45"
## [1] 2
## [1] "CT"
## [1] "Tree 46"
## [1] 2
## [1] "CT"
## [1] "Tree 47"
## [1] 2
## [1] "CT"
## [1] "Tree 48"
## [1] 2
## [1] "CT"
## [1] "Tree 49"
## [1] 2
## [1] "CT"
## [1] "Tree 50"
## [1] 2
## [1] "CT"
## [1] "Tree 51"
## [1] 2
## [1] "CT"
## [1] "Tree 52"
## [1] 2
## [1] "CT"
## [1] "Tree 53"
## [1] 2
## [1] "CT"
## [1] "Tree 54"
## [1] 2
## [1] "CT"
## [1] "Tree 55"
## [1] 2
## [1] "CT"
## [1] "Tree 56"
## [1] 2
## [1] "CT"
## [1] "Tree 57"
## [1] 2
## [1] "CT"
## [1] "Tree 58"
## [1] 2
## [1] "CT"
## [1] "Tree 59"
## [1] 2
## [1] "CT"
## [1] "Tree 60"
## [1] 2
## [1] "CT"
## [1] "Tree 61"
## [1] 2
## [1] "CT"
## [1] "Tree 62"
## [1] 2
## [1] "CT"
## [1] "Tree 63"
## [1] 2
## [1] "CT"
## [1] "Tree 64"
## [1] 2
## [1] "CT"
## [1] "Tree 65"
## [1] 2
## [1] "CT"
## [1] "Tree 66"
## [1] 2
## [1] "CT"
## [1] "Tree 67"
## [1] 2
## [1] "CT"
## [1] "Tree 68"
## [1] 2
## [1] "CT"
## [1] "Tree 69"
## [1] 2
## [1] "CT"
## [1] "Tree 70"
## [1] 2
## [1] "CT"
## [1] "Tree 71"
## [1] 2
## [1] "CT"
## [1] "Tree 72"
## [1] 2
## [1] "CT"
## [1] "Tree 73"
## [1] 2
## [1] "CT"
## [1] "Tree 74"
## [1] 2
## [1] "CT"
## [1] "Tree 75"
## [1] 2
## [1] "CT"
## [1] "Tree 76"
## [1] 2
## [1] "CT"
## [1] "Tree 77"
## [1] 2
## [1] "CT"
## [1] "Tree 78"
## [1] 2
## [1] "CT"
## [1] "Tree 79"
## [1] 2
## [1] "CT"
## [1] "Tree 80"
## [1] 2
## [1] "CT"
## [1] "Tree 81"
## [1] 2
## [1] "CT"
## [1] "Tree 82"
## [1] 2
## [1] "CT"
## [1] "Tree 83"
## [1] 2
## [1] "CT"
## [1] "Tree 84"
## [1] 2
## [1] "CT"
## [1] "Tree 85"
## [1] 2
## [1] "CT"
## [1] "Tree 86"
## [1] 2
## [1] "CT"
## [1] "Tree 87"
## [1] 2
## [1] "CT"
## [1] "Tree 88"
## [1] 2
## [1] "CT"
## [1] "Tree 89"
## [1] 2
## [1] "CT"
## [1] "Tree 90"
## [1] 2
## [1] "CT"
## [1] "Tree 91"
## [1] 2
## [1] "CT"
## [1] "Tree 92"
## [1] 2
## [1] "CT"
## [1] "Tree 93"
## [1] 2
## [1] "CT"
## [1] "Tree 94"
## [1] 2
## [1] "CT"
## [1] "Tree 95"
## [1] 2
## [1] "CT"
## [1] "Tree 96"
## [1] 2
## [1] "CT"
## [1] "Tree 97"
## [1] 2
## [1] "CT"
## [1] "Tree 98"
## [1] 2
## [1] "CT"
## [1] "Tree 99"
## [1] 2
## [1] "CT"
## [1] "Tree 100"
## [1] 2
## [1] "CT"
\end{verbatim}

\begin{verbatim}
## [1] 2201  100
\end{verbatim}

The \(tau_1^{PRED}\) from causal match is 0.0305319 and from causal
forest 0.0263373. Their respective errors are 2.008498 and 3.37339. The
NPE is 6.1686766.


\end{document}
